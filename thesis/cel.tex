\chapter{Cel pracy}
Cel pośredni pracy stanowiła symulacja rozplatania pojedynczego modułu kinazy tytyny przy różnych szybkościach rozciągania za pomocą programu Gromacs. Jako strukturę wyjściową wzięto strukturę krystalograficzną białka o symbolu 1TKI z bazy danych RCSB PDB \cite{pdbdb}. Wyniki symulacji zostały porównane z wynikami literaturowymi. 

Celem bezpośrednim było opracowanie uniwersalnej, niezależnej od badanej cząsteczki procedury numerycznej mającej za zadanie dopasowanie do danych z symulacji rozciągania krzywej teoretycznej z modelu mikroskopowego Hummera--Szabo\cite{Hummer_Szabo_2003}. Dopasowanie to pozwoliło na odzyskanie parametrów kinetycznych oraz porównanie wyników z danymi z eksperymentów AFM.

Podczas symulacji prowadzono także testy obliczeniowe klastra, dzięki czemu możliwa stała się analiza wydajności oraz ocena korzyści płynących z dalszej rozbudowy klastra.

Praca skaładała się z kilku etapów:
\begin{itemize}
\item Budowa klastra obliczeniowego
\item Symulacje
\begin{itemize}
\item Dobór parametrów symulacji
\item Równowagowanie
\item Właściwe obliczenia
\end{itemize}
\item Analiza wyników
\begin{itemize}
\item Odzyskanie sił rozplatania z wyników symulacji
\item Dopasowanie modelu do danych i odzyskanie parametrów kinetycznych
\end{itemize}
\item Wnioski
\end{itemize}

Praca ta ma także stanowić krótkie omówienie procedury przeprowadzania obliczeń programem Gromacs, kilku najważniejszych parametrów i problemów, które może napotkać początkujący użytkownik, tak aby zdobyta wiedza stanowiła w przyszłości punkt wyjścia dla będących zainteresowanymi rozpoczęciem pracy z tym programem.