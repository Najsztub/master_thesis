\chapter{Wstęp}

\parskip=0pt
Wraz z rozwojem techniki aparaturowej pojawiły się nowe możliwości badania procesów molekularnych na poziomie pojedynczej cząsteczki. Jednym z takich narzędzi jest mikroskop sił atomowych (AFM), który zostanie omówiony w dalszej części pracy. Nasuwa się pytanie czy możliwym jest uzyskanie wyników zbliżonych do danych eksperymentalnych tylko za pomocą symulacji komputerowych?

Temat tej pracy magisterskiej obejmuje próbę ekstrapolacji wyników symulacji rozplatania białka. Przedstawiona jest także próba odzyskania parametrów kinetycznych procesu rozplatania na podstawie symulacji. Obliczenia były prowadzone za pomocą pakietu do symulacji dynamiki molekularnej \textsc{Gromacs}\cite{Gromacs_manual}. Program ten był już wcześniej używany na Pracowni Dynamiki Procesów Molekularnych~\cite{marek}, jednak ze względu na ograniczenia wynikające z długiego czasu obliczeń oraz ilości miejsca potrzebnego na gromadzenie danych używanie go przy zasobach dostępnych w 2007 roku było kłopotliwe. Wraz z rozwojem tego programu oraz z zakupem coraz to nowszych komputerów pojawiła się szansa na przezwyciężenie tych ograniczeń.

W pracy przedstawione zostaną wyniki symulacji rozciągania pojedynczego modułu tytyny 1TKI oraz próba połączenia tych wyników z dostępnymi danymi uzyskanymi w wyniku poprzednich eksperymentów. Ponadto przedstawiona zostanie analiza wydajności klastra oraz perspektyw jego rozbudowy.

Na wstępie zostaną zaprezentowane metody eksperymentalne AFM oraz omówione programy do symulacji dynamiki molekularnej. Następnie przedstawiony zostanie proces przygotowania oraz zbierania wyników gdzie opisano szczegółowo metodologię pracy. Osobny rozdział poświęcony jest analizie danych oraz procedurze dopasowania do nich krzywej teoretycznej.