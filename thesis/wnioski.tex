\chapter{Wnioski}

Na podstawie przeprowadzonych obliczeń można wywnioskować, że symulacje rozplatania białek o znanej strukturze mogą być pomocne przy odzyskiwaniu parametrów kinetycznych. Należy jednak zawsze pamiętać, że metoda ta może być tylko wspomaganiem eksperymentu. Ostatecznym potwierdzeniem zawsze musi być emipryczny, dający się powtórzyć wynik doświadczalny, a nie tylko teoretyczny. 

Dokładność uzyskanych parametrów kinetycznych w ogromnej mierze zależy od ilości oraz jakości uzyskanych wyników symulacji. Stosując dużą próbę statystyczną uzyskujemy lepsze wyniki. Pojedyncze wartości sił rozplatania czy też zerwania stanowią zbyt słabą przesłankę co do prawdziwości odzyskanych parametrów kinetycznych. Ponadto w celu poprawnego dopasowania ogromny wpływ ma także duża rozpiętość badanych szybkości. Im więcej, tym lepiej. Badana zależność sił rozplatania od prędkości rozciągania ma jednak charakter logarytmiczny. Pociąga to za sobą wykładniczy wzrost czasu potrzebnego do symulacji, a co za tym idzie czasu obliczeń. Z tego powodu zachodzi potrzeba zakupu coraz to wydajniejszego sprzętu. 

Pojedyncze wyniki, tak jak w tym przypadku powodują powstawanie dużego błędu z powodu małej liczebności grupy. Oznaczane parametry, jak na przykład stała rozplatania $k_{0}$ charakteryzują się dużą czułością i nawet błąd kilkudziesięciu pN w oznaczaniu siły rozplatania może oznaczać różnice wyznaczonej stałej szybkości sięgającą kilku rzędów wielkości. Nie tyle rozwiązaniem, co ograniczeniem tego problemu jest wspomniane wcześniej zwiększenie próby, co pociąga za sobą jednak zwiększenie czasu obliczeń. Za pociesznie może jednak służyć fakt, że rozrzut sił rozplatania maleje wraz ze spadkiem prędkości. Z drugiej strony jednak dokładność uzyskanych parametrów kinetycznych zależy przede wszystkim od wartości dla mniejszych prędkości.

Przy odzyskanych parametrach kinetycznych obliczone na ich podstawie siły dla eksperymentalnego zakresu prędkości znacząco odbiegają od tych wyznaczonych doświadczalnie. Wynika to z czułości algorytmu dopasowania na wyniki symulacji. Ponadto ekstrapolacja sił rozplatania dla zakresu eksperymentalnego jest bardzo czuła na zmiany parametrów kinetycznych, które dla zmiany sił rozplatania o 10\% mogą przyjmować wartości różniące się od siebie nawet o kilka rzędów wielkości. Dlatego też metoda ta wymaga dalszego doskonalenia. Przy większej liczbie wyników symulacji dla danej szybkości pomocne może być uwzględnienie odchyleń standardowych przy metodzie najmniejszych kwadratów poprzez stosowanie ważenia. Wymaga to jednak większej ilości wartości sił rozplatania dla danej szybkości.

Budowa klastra na zwykłej sieci o przepustowości 1 Gbps opartej na zwykłym switchu przy maszynach wieloprocesorowych, kiedy to na jednej maszynie może być uruchomionych aż 8 niezależnych procesów jest nieefektywna. Ze względu na niewielki wzrost wydajności przy użyciu wszystkich 16 wątków rozsądniejsze wydaje się prowadzenie obliczeń równolegle na dwóch maszynach. Wynika to z prostego faktu, że komunikacja pomiędzy procesami uruchomionymi na jednej płycie głównej jest nieporównywalnie szybsza niż pomiędzy tymi samymi procesami połączonymi przez sieć. W zastosowaniu znajdują się specjalne szybsze łącza typu Infiniband czy Myrinet, które pozwalają na uzyskanie lepszej skalowalności. 

Nie można także zapominać o wiedzy potrzebnej do prawidłowej obsługi programów do symulacji dynamiki molekularnej. W przypadku programu Gromacs opanowanie parametrów służących do kontroli symulacji zajęło około rok bazując przede wszystkim na znalezionych w internecie samouczkach oraz listach mailingowych służących do wymiany myśli pomiędzy użytkownikami. Dlatego też dobór takich opcji jak czas równowagowania, sprotonowanie, zespół, typ barostatu i termostatu ma kluczowy wpływ na wynik. Nieznajomość parametrów, a co za tym idzie nieumiejętność ich kontroli może prowadzić do sytuacji, kiedy to wynik symulacji nie ma nic wspólnego z rzeczywistością. Nawet będąc pewnym swoich umiejętności zawsze należy podchodzić do wyniku symulacji z dużą rezerwą, ponieważ jest to tylko wynik teoretyczny. 

Symulacje dynamiki molekularnej stanowią ogromną wartość jako narzędzie do badania i wizualizacji procesów odbywających się w skali pojedynczej cząsteczki. Dzięki temu mamy możliwość wglądu na zachowanie oraz rolę poszczególnych atomów w trakcie danego procesu. 